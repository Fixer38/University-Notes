\chapter{Pipeline améliorations et problèmes}
\lecture{5}{Lundi 14 Octobre 2019 13:10}{Pipeline}
\subsection{Incidence sur la durée d'exécution du programme}
\begin{enumerate}
  \item La durée d'exécution d'une instruction n'a pas été modifiée mais la durée d'\textbf{exécution totale du programme entier (suite d'instructions) a été fortement réduite}
  \item Cela a été réalisé en augmentant le nombre d'instructions terminées par unité de temps
  \item Le pipeline rend plus efficace l'utilisation des ressources du CPU en faisant travailler \textbf{simultanément ses différentes sous-unités}
\end{enumerate}

\subsection{Améliorer les performances}
\begin{enumerate}
  \item En passant d'un processeur single cycle à un processeur ayant une profondeur de pipeline de 4, on observe une amélioration d'un facteur 4
  \item Que se passerait-il si la profondeur du pipeline passait à 5, 6, 7... avec des durées d'étapes intermédiaires de plus en plus courtes? L'instruction va durer le même temps, des subdivisions plus petites seront présentes dans les étapes (fetch-decode-execute-write)
  \item + d'étapes -> + d'instructions en cours d'exécution dans le pipeline -> + d'intructions traités par secondes
  \item Avec un découpage en 8 étapes, on obtient 1 instruction terminée toutes les 0,5 ns 
  \item Completion rate = 2 instr/s
\end{enumerate}

\subsection{Limites du pipeline en profondeur 4}
\begin{enumerate}
\item En 8ns, le pipeline a permis d'exécuter 5 instructions car \textbf{certaines cases ne sont pas remplies au début du programme}
\item En single cycle aurait exécuté 2 instructions dans le même temps
\item 5 au lieu de 2 ne se représente pas vraiment comme un gain de 4
\item 5 instructions/8ns donne un completion rate de 0,625 inst/ns ce qui ne représente pas le 1 inst/ns
\end{enumerate}

\subsubsection{analyse}
\begin{enumerate}
\item Pour que le gain du au pipeline devienne appréciable, il faudra que le nombre d'instructions dans le programme soit suffisant
\item Exemple: sur 1000ns, un processeur a pipeline de profondeur 4 aura exécuté 996 instructions au lieu des 250 sans pipeline
\item soit un gain de 3,984 proche de 4
\item 4 est en réalité \textbf{un maximum théorique et 3,984 est la moyenne réelle}
\end{enumerate}

\subsubsection{Décrochage du pipeline}
\begin{enumerate}
\item Il peut arriver que des étapes du pipeline nécessitent plus d'un cycle horloge
\item Des "bulles" apparaissent et voyagent alors dans le pipeline, réduisant ainsi ses performances -> Repassage à une case vide -> \textbf{L'instruction prend donc plusieurs cycles machine pour s'exécuter}
\item La latence d'une instruction peut traduire le temps réel qu'elle met pour traverser le pipeline en prenant en compte les bulles
\end{enumerate}

\subsection{Valeurs réelles}
\begin{enumerate}
\item En pratique un processeur cadencé à 3GHz peut facilement attendre 50 à 120ns des données en provenance de la mémoire centrale
\item Une centaine de ns représentent à ces cadences, quelques milliers de cycles d'horloge 
\item Et cela pour un seul accès mémoire. Or il s'en produit des millions lors du déroulement d'un programme...
\item 120ns = 120⁻⁹ sec 3GHz = 3⁹HZ
\item Nombre de cycles d'horloges en 120ns -> t=1/f -> t=1/3⁹ = 3,33333⁻¹⁰ -> 120⁻⁹/t = 360.0
\end{enumerate}

\subsection{CCL sur le décrochage de pipeline}
\begin{enumerate}
\item Les bulles doivent absolument être évitées au maximum
\item Elles sont généralement dues à des accès trop lents à la mémoire centrale
\end{enumerate}

Solutions:
\begin{enumerate}
\item Mémoire cache
\item D'autres causes et pistes de solutions associées existent
\end{enumerate}

\subsection{Limites du pipeline}
\begin{enumerate}
\item Il est difficile de découper une instruction en plusieurs étapes équivalentes qu'on voudrait
\item Toujours étapes finiront par être plus longues que certaines (car indivisibles)
\item Ces étapes plus longues influencent le rythme de l'horloge
\end{enumerate}
En fait on peut même concevoir qu'une instruction réalisée dans un pipeline peut prendre plus de temps pour s'exécuter que dans un single cycle, exemple:
\begin{enumerate}
\item Une instruction prend 4ns pour se réaliser en single cycle
\item On la sépare en 4 étapes dont les durées respectives sont w,x,y, z ns (proche de 1ns mais pas parfaitement identiques)
\item x = la \textbf{plus longue qui impose le rythme} (x > 1ns) -> temps total = 4 fois x > 4ns
\end{enumerate}
L'horloge est une limite
\begin{enumerate}
\item On ne peut pas atteindre l'horloge théorique idéale: horloge sans pipeline/ profondeur du pipeline
\item l'horloge réelle est limitée à une cadence maximale -> surchauffe -> dégradation
\item A une même cadence limite, un processeur avec une profondeur plus élevée mettra alors plus de temps pour se remplir
\end{enumerate}

\subsubsection{exemple}
\begin{enumerate}
\item un pipeline en 4 stages peut être plus rapide qu'une en 8 stages dû aux bulles -> étapes vont durer plus longtemps que la fréquence du CPU -> multiplication des bulles -> dégradation du nombre d'instructions sorties par cycle d'horloge
\end{enumerate}

\subsubsection{Problèmes}
\begin{enumerate}
\item Cout financier -> plus de transistors
\item L'architecture à pipeline est complexe et requiert des composants supplémentaires
\item Ces composants ont un coût
\item Dégagement de chaleurs par les circuits
\end{enumerate}
