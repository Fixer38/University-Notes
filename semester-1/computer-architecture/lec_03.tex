\chapter{CPU}
\lecture{3}{Lundi 30 Septembre 2019 13:10}{}
\section{CPU}
\subsection{L'horloge}
\begin{enumerate}
  \item Étapes vues jusqu'ici sont réalisées au rythme de 1 pulsation d'horloge
  \item L'horloge est in signal cadencé à vitesse (fréquence) fixe par un dispositif dédié
  \item Idéalement, un cycle fetch-code-execute doit être réalisé en une impulsion d'horloge
  \item A un certain moment les données ne sont pas bonnes, la fréquence d'horloge sert à synchroniser toutes les données pour s'exécuter proprement
\end{enumerate}

\section{Les instructions de saut inconditionnel}
\begin{enumerate}
\item Unconditional branch = jump \#target
\item Permet de continuer l'exécution d'un programme à un autre endroit dans la mémoire
\item La cible du saut (target) est une valeur immédiate (\#500) ou une adresse stockée dans un registre (\#D ou \#(D+103))
\item Charge une valeur dans le PC (program counter)
\item Attention: registre et bus d'adresse doivent être de même longueur (en bits) sinon utilisation de plusieurs registres si plus long -> création de banque de données (bits qui ne changent pas)
\end{enumerate}

\section{Les instructions de saut conditionnel} 
\begin{enumerate}
\item Conditional Branch = jumpx \#target
\item Le PC est chargé si la condition est remplie
\item La condition est souvent un bit du PSW
\item Le PSW a plusieurs bits de "résultats d'opérations"
\item 1(true) ou 0(false) sera écrit dans le PSW selon une condition bien précise 
\item Pour vérifier que la condition est remplie il suffira donc de vérifier le bit approprié dans PSW(flags)
\end{enumerate}
\subsection{exemple}
\begin{lstlisting}
sub A, B, C ; soustraire le nombre contenu dans le registre 1 du nombre contenu dans le registre B et stocker le résultat dans C
jumpz #106  ; vérifier le PSW et si le résultat de l'instruction précédente est 0, sauter à l'instruction à l'adresse #106. Si le résultat est != 0, continuer à la ligne 20
add A, B, C ; Additioner les nombres contenus dans les registres A et B et stocker le résultat en C
\end{lstlisting}
Incrementation de 2 sur le nombres d'adresse car chaque instruction codé sur 2 bits

\subsection{exemples de jump}
\begin{enumerate}
\item jumpn
\item jumpo
\end{enumerate}

\subsection{label}
Plus facile d'utiliser des labels à la place des adresses mémoires (absolues ou relatives)
\begin{lstlisting}
sub A, B, A ; soustrait A à B, résultat dans A
jumpz LBL1 ; jump à LBL1 si résultat est égal à zéro
add A, 15, A ; ajouter 15 à A résultat dans A
store A, #(D + 16) ; stocker A dans la cellule #D+16
LBL1: ; définition du label LBL1
  add A, B, B ; ajouter A à B stocker résultat dans B
store B, #(D+16) ; stocker B dans la cellule #D+16
\end{lstlisting}

