\chapter{Pipelined Execution}
lecture{4}{Lundi 07 octobre 2019 8:30}{Pipelined Execution}
\section{Rappel du cycle de base}
\begin{enumerate}
  \item Fetch 
  \item Decode
  \item Execute
\end{enumerate}
Execute divisé en 3 se passe dans l'ALU, elle ne traite que des infos provenant de registres
\begin{enumerate}
  \item Read 
  \item Add 
  \item Write
\end{enumerate}

\section{Remarques}
La plupart des micro processeurs actuels traitent
\begin{enumerate}
  \item 3a(read) et 3b(add) sont comme un groupe
  \item La sous-étape 3c(write) traité séparément
\end{enumerate}

\section{4 phases d'exécution}
\begin{enumerate}
  \item Fetch
  \item Decode
  \item Execute
  \item Read and Add
  \item Write
\end{enumerate}

\subsection{Rappel de l'exécution}
3 bus sont entre la RAM et le CPU
\begin{enumerate}
\item le bus d'adresse, regroupe les adresses adressables
\item le bus de contrôle, Commandes de lecture et écriture, contient le bus d'horloge pour dire que la donnée est valide (problème de fils plus long, exemple: changer 3 bits d'un coup ne se fait pas en même temps) 
\item le bus de données, contient et transporte les données
\end{enumerate}
Dans l'AR1 -> 4 registres ABCD et PC et l'IR
\begin{enumerate}
  \item il démarre et va exécuter la première instruction -> envoi de la première adresse au PC
  Les adresses de démarrage sont réservées par le micro processeur
  \item envoi de l'adresse jusqu'au bus d'adresse, validation de l'adresse -> je lis ce qui se trouve à l'adresse -> envoi de la cellule mémoire contenue dans la RAM vers le bus de données -> arrive dans l'IR
  \item Le PC va être incrémenté après avoir fini de lire une adresse -> lecture de l'adresse suivante -> envoi dans le bus d'adresse -> lecture dans le bus de contrôle -> cellule mémoire arrive dans le bus de données 
  \item Envoi dans l'IR -> exécution de l'instruction (l'IR est de 2 octets de long, peut stocker une instruction longue de 10 bits)
  \item exemple: instruction avec un résultat dans C (add A,B, C - load A, C)

\end{enumerate}

\section{Program Execute et Completion Rate}
Program Execution Time = Number of instructions in program / Instruction Completion Rate
Prise de valeur moyenne, chaque instruction prend un temps différent

\subsection{Completion Rate}
\begin{enumerate}
  \item Nombre d'instructions terminées par l'unité de temps -> instr/ns
  \item Ne pas confondre avec le temps d'exécution d'une instruction en ns
  \item Dans un CPU sans \textbf{pipeline} le rapport entre les 2 est de type inversement proportionnel (l'une est égale à l'inverse de l'autre)
\end{enumerate}

\subsection{pipeline}
\begin{enumerate}
  \item Ce système permet d'augmenter le Completion Rate sas modifier le temps d'exécution du temps d'une instruction (fin de la relation inversement proportionnel)
  \item Certaines instructions mettront plus de temps à s'exécuter avec un CPU à pipeline que sans, celles-ci ne peuvent pas se mettre dans la pipeline
\end{enumerate}

\subsubsection{Augmenter les performances}
\begin{enumerate}
\item 2 pistes
\begin{itemize}
\item Réduire le nombre d'instructions du programme
\item Augmenter le completion rate
\end{itemize}
\item Nombre d'intructions du programme est fixé pour l'exemple
\end{enumerate}

\subsubsection{analogie fabrique de voiture}
La chaine de production est composée de 5 équipes distinctes spécialisées dans une seule tâche
\begin{enumerate}
  \item Fabriquer le chassis, il attend la fin de la chaine pour commencer un nouveau chassis
  \item Mettre le moteur sur le chassis
  \item Mettre les portes, ailes et le capot
  \item Attacher les roues
  \item Peindre le véhicule
  \item Chaque équipe fini par attendre la fin de la chaîne pour recommencer à construire une voiture
\end{enumerate}

\subsubsection{Seconde approche avec pipeline}
Si chaque équipe ne s'arrête jamais de travailler et recommence leur travail spécifique au lieu d'attendre que la chaîne soit finie pour recommencer à travailler

\subsubsection{Gain}
\begin{enumerate}
  \item Dès la 6ème heure, on passe de 1 véhicule toutes les 5h à 1 véhicule par heure
  \item Sans changer le temps total pour fabriquer un véhicule on améliore d'un facteur 5 les performances
  \item Pour peu que la chaine de fabrication reste pleine
\end{enumerate}

\subsection{Processeurs single cycle}
\begin{enumerate}
  \item Single Cycle = absence de pipeline
  \item 1 cycle de vie d'instruction = 1 cycle d'horloge = fetch - decode - execute - write
  \item les 4 phases vont être exécutées sur une impulsion d'horloge (besoin d'un cycle d'horloge car changement de bits -> attente d'un certain temps, unité de temps = impulsion horloge)
  \item Chaque phase va attendre que le cycle soit terminé pour recommencer à travailler
\end{enumerate}

\subsubsection{Observations}
\begin{enumerate}
  \item Pour améliorer le Completion rate, il faut cadencer l'horloge de + en + vite
  \item Sans dépasser le temps maximum que prend une instruction pour s'exécuter complètement
  \item Illustration précédente: un programme de 4 instructions prendra 16ns, soit un completion rate de 0,25 instr/ns
\end{enumerate}

\subsection{Processeurs avec Pipeline}
exemple
\begin{enumerate}
  \item Avec une profondeur de pipeline de 4
  \item Après la première phase, durant la deuxième phase un autre fetch s'exécute
  \item chaque phase n'attend pas la fin du cycle, elle travaille directement sur la prochaine instruction
  \item exemple: Fetch 1ère instruction -> decode 1ère instruction + fetch 2ème instruction -> Execute 1ère instru + decode 2ème instru + fetch 3ème instru
  \item Toutes les 4 nanosecondes une instruction est effectuée est toutes les nanosecondes une instruction sort
\end{enumerate}

\subsubsection{Observations}
\begin{enumerate}
  \item Dès le début de la 5ème ns, une instruction est exécutée toutes les ns
  \item Dans cette exemple un programme de 4 instructions prendra 8ns, completion rate de 1 instr/ns -> gain de 4 sur le nombre d'instructions traitées par ns(car 4 phases)
\end{enumerate}

\subsection{L'incidence sur l'horloge}
\begin{enumerate}
\item L'horloge ne cadence plus une instruction complète mais les phases d'instructions
\item -> Les pulsations sont bien plus courtes (exemple: 4ns -> 1ns)
\item Ceci implique qu'une instruction ne doit plus nécessairement être finie à la fin de chaque cycle d'horloge (ex: avant le début de la 5ème ns dans l'illustration précédente)
\end{enumerate}
