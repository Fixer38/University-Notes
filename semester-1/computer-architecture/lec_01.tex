\chapter{Concepts de base}
\lecture{1}{Mardi 17 September 2019 8:30}{Instructions, registre et ALU}
\section{Instructions}
\begin{enumerate}
\item Une série d'instructions arrivent au cœur de l'ordinateur.
Traitées par le cpu pour avoir un résultat.
Exemple:
\begin{itemize}
  \item 1. Instruction: addition
  \item 2. Données: 1 et 2 
  \item 3. Résultat: 3
\end{itemize}
\textbf{Notion de flux continu}

\item Explications exemple: 
\begin{itemize}
  \item 1. Lire les nombres
  \item 2. Modifier les nombres
  \item 3. Écrire les nombres
\end{itemize}
Conséquence sur le design des composants -> Accélérer les actions.\bigskip

\item En détails:
\begin{itemize}
  \item1. Lecture/Écriture -> nécessite une zone modifiable pour le stockage des données

  \item2. Modifier -> Nécessite une ALU pour modifier les données. Celle-ci faisant partie du CPU.
Partie du CPU qui execute le calcul.

 \item 3. Manipuler les données -> nécessite un chemin, les bus.
 \begin{itemize}
    \item 1. Data Bus -> ensemble des fils qui traitent les données.
    \item 2. Instruction bus -> ensemble des fils qui transportent les instructions.
  L'ensemble des fils -> bus.
  \end{itemize}
  \item 4. Résultat -> les données résultantes vont être renvoyées dans la zone mémoire. (écrire)
\end{itemize}
\end{enumerate}
\section{Registres}
\begin{enumerate}
\item Exemple de réflexion architecturale: 
Les registres proches de l'ALU ont de meilleurs performances dans le traitement de données. -> Registres internes au CPU.
Registre -> petit espace de stockage, rapide incorporé au CPU.

\bigskip
\item Exemple de l'addition: A+B=C
\begin{itemize}
  \item 1. Obtenir depuis les registres sources (A et B)
  \item 2. Additionner les nombres.
  \item 3. Placer le résultat dans le registre de destination (C) -> Écrase ce qui se trouvait dans le registre(0 et 1) -> les registres ne sont jamais vides.
\end{itemize}
\item{\textbf{La mémoire RAM}}
\begin{itemize}
  \item 1. Les registres sont petits -> contiennent peu de données.
  \item 2. Données utiles -> déplacées vers les registres(accumulateur).
  \item 3. Le reste -> dans \textbf{main memory} composée de mémoire de type \textbf{RAM(Random access memory) -> Accès RW(Lecture/Écriture)}, supprimée si plus de courant, besoin de \textbf{rafraichir la mémoire} pour garder les infos.).
  \item \textbf{RAM statique} -> conserve la info sans rafraichissement.
\end{itemize}
Schema: Main memory: Instructions + données(lire)
CPU = ALU + registres (modifier) in memory bus
Écrire: Résultat -> données in main memory

\item Exemple Addition: A+B=C
\begin{itemize}
  \item1. Charger les deux opérandes (A et B) depuis la mémoire centrale (main memory) vers deux registres sources.

  \item 2. Additionner:
  \begin{itemize}
    \item a. Lire le contenu des registres A et B
    \item b. Additionner les contenus des registres A et B
    \item c. Écrire le résultat dans le registre C
  \end{itemize}
\end{itemize}
\item \textbf{Le programme: code stream}
Le code -> série d'instructions de commandes qui dépend du type de CPU(X86, ARM...) -> précise ce que la machine doit faire. -> exemple: Entrée reset.

Exemple: 
\begin{itemize}
  \item1. Instruction \textbf{load} pour charger les nombres depuis la mémoire vers les registres
  \item 2. Instruction \textbf{add} pour que le ALU réalise l'addition.
  \item 3. Instruction \textbf{store} pour que l'ordinateur place le résultat en mémoire.
\end{itemize}
\item \textbf{Catégorie d'instructions}
\begin{itemize}
  \item 1. Arithmétique: add, sub, mul, div
  \item 2. Accès mémoire: load, store
  \item 3. Branchement (jump)
  \item 4. Logique (conditions): and, or, not
\end{itemize}
\item \textbf{Architecture basique et format d'instructions}
Basé sur une machine fictive: \textbf{AR1} composé de:
\begin{itemize}
  \item 1. 1 ALU
  \item 2. 4 registres: A, B, C, D
  \item 3. 256 cellules mémoires addressable de 0 à 255
\end{itemize}
\begin{enumerate}
\item \textbf{Format des instructions:} 
\begin{lstlisting}
instruction source1, source2, destination
\end{lstlisting}
Exemple: 
\begin{lstlisting}
add A, B, C 
#Passage dans le compilateur transforme les nbs binaires.
\end{lstlisting}
\item \textbf{Format des instructions d'accès mémoire}
\begin{lstlisting}
Instruction source, destination
#exemple:
load \#12, A
#exemple d'un programme:
load \#12, A
load \#13, B
add A, B, C
store C, \#14
#Les donnees en source ne changent pas.
\end{lstlisting}
\item Limites de l'exemple: 
Comment connaitre le contenu des cellules \#12 et \#13.
\end{enumerate}
\item \textbf{Valeurs immédiates et jeux sur les addresses}
\begin{enumerate}
\item Utilisation d'une valeur à la place d'une adresse.
\begin{lstlisting}
add, A, 2, A 
# Ajouter la valeur 2 au contenu du registre A et ecrire le result dans A (en ecrasant ce qui s'y trouvait)
#Utilisation du symbole # pour aller chercher une adresse.
\end{lstlisting}
\item 2ème programme (utilisation de D sans savoir sa valeur):
\begin{lstlisting}
load #D, A ;Contenu de #D = 12, valeur de la cellule 12 dans A
load #13, B ;Contenu de la cellule 13 dans B
add A, B, C ;Add A+B -> result dans C
store C, #14 ;Stocker C dans cellule 14
\end{lstlisting}
\textbf{Le \# pointe vers la case mémoire contenue dans le registre mentionné}
\item 3ème programme:
\begin{lstlisting}
load #11, D ;Contenu de la cellule 11 dans D(pointeur)
load #D, A ;Contenu de la cellule D dans A (cellule 11) 
load #13, B ;Contenu de la cellule 13 dans B
add A, B, C ;Add A+B -> result dans C 
store C, #14 ;Stocker C dans cellule 14
\end{lstlisting}
\end{enumerate}
\item Résumé:
\begin{itemize}
  \item1. Mettre une valeur dans un registre
  \item2. Récupérer cette valeur en tant qu'adresse de cellule mémoire.
  \item3. Lire le contenu de cette cellule.
  \item4. Charger ce contenu dans un registre.
  \item5. La mémoire est divisée en segments(bits consécutifs):
  \begin{itemize}
    \item1. Certains stockent des données.
    \item2. D'autres stockent du code.
  \end{itemize}
\end{itemize}
\item\textbf{Adressage relatif = adresse de base + offset}
Exemples:
\begin{lstlisting}
load #(D+108), A
store B, #(D+108)
\end{lstlisting}
\item \textbf{Les adresses mémoires et les nombres entiers}: sont stockés dans les mêmes registres appelés: \textbf{general purpose registers(GPRs)}

AR1: A,B,C,D = GPRs.
\item \textbf{Autres registres:}
Registre de \textbf{statut \& registre de contrôle}.

Schema + bus de contrôle, tout est addressable.
\end{enumerate}
