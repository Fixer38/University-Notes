\chapter{Processus et threads}
\lecture{3}{Lundi 14 Octobre 2019 8:30}{Processus et threads}
\section{Processus}
Le premier processus lancé est l'OS
Chaque programme lancé est un processus

\subsection{Les processus}
\begin{enumerate}
  \item Programme = ligne de code, il a des variables propres à lui, des appels aux fonctions, son contexte
  \item L'OS va donner le contexte au programme et toutes ses variables qui vont avec
  \item Processus = code en exécution
\end{enumerate}

\subsection{Types de processus}
\begin{enumerate}
  \item Les processus qui tournent en background
  \item Les programmes interactifs
\end{enumerate}

\subsection{Création des processus}
\begin{enumerate}
  \item Initialisation système
  \item Processus parent
  \item Requête
  \item Batch -> processus qui a peu d'interaction avec l'utilisateur et se lance à un moment précis (ex: backup)
\end{enumerate}

\subsubsection{Sous Unix/Linux}
\begin{enumerate}
  \item Unix: base processus
  \item Processus 0
  \item Processus 1: init
  \item /etc/iinitab -> iinitab = table des processus à lancer
  \item dameon = processus qui attendent un event (ex: inetd -> gestion des ports internet)
  \item fork -> programme qui lance un autre programme (copie de lui-même)
  \item gettty = programme utilisateur (prêt à recevoir un login)
  \item process login -> A la connection d'un utilisateur -> va voir dans le fichier passwd pour le mdp -> ok -> lance un shell (BASH)
\end{enumerate}

\subsubsection{Sous Windows}
\begin{enumerate}
  \item Une fonction WIN32 -> CreateProcess se charge à la fois de la création du nouveau processus et de sa personalisation
  \item Des paramètres lui sont fournis
  \item Démarre les programmes mentionnés
\end{enumerate}

\subsection{La fin d'un processus}
\begin{enumerate}
  \item Normal exit -> voluntary
  \item Error exit -> voluntary
  \item Fatal error -> unvoluntary
  \item Killed by another process -> unvoluntary (commande: kill -> kill -9) -> signal qui est envoyé au processus (kill, siguser...)
\end{enumerate}
exemple: division par 0 -> impossible -> détecter par l'OS -> arrêt du programme

\subsubsection{Processus normal exit}
\begin{enumerate}
  \item Fin de programme
  \item Intervention utilisateur
  \item Linux: appel exit
  \item WIndows: ExitProcess
  \item Libération des ressources
  \item PCB effacé (Process control block)
\end{enumerate}

\subsection{État des processus}
\textbf{QUESTION EXAMEN}
\begin{enumerate}
\item Lorsque un programme est réveillé il se retrouve -> prêt
\item Passé de l'état prêt à Élu -> élection (dépend de l'ordre d'importance du programme)
\item Élu -> le programme à la main sur le CPU
\item CPU veut donner du temps aux autres programmes -> suspend le programme qui avait la main
\item exemple: Le process demande un nom à l'utilisateur -> le process est bloqué -> le CPU le libère
\item si autre process est prêt -> le CPU l'élu -> le process à la main sur le CPU 
\item lors de l'input de l'utilisateur -> process débloqué -> repassage à l'état prêt
\end{enumerate}

\subsubsection{États particulier}
\begin{enumerate}
\item Initialisation
\item Terminé
\item Zombie -> toujours des programmes fils -> ne sait pas ce qu'il fait là
\item Swappé -> process qui est envoyé vers la swap pour travailler
\item Préempté -> processus qui arrête de travailler pour donner du temps de process aux autres
\item Utilisateur
\item Noyau -> fonction root
\end{enumerate}

\subsubsection{préemption}
\begin{enumerate}
\item Process qui passe la main à un autre dépendant de sa priorité
\item Transition de l'état Élu à l'état prêt -> permet aux programmes de tourner en "même temps" -> Distribution du CPU aux différents processus 
\end{enumerate}

\subsection{Les processus et les threads}
\begin{enumerate}
\item Première partie de la mémoire lue (données initialisées) -> section .text
\item Variables non initialisées chargées -> section .data
\item Les variables de travail situées dans les fonctions -> dans la stack
\end{enumerate}
Initialisation d'une fonction dans la \textbf{stack}
\begin{enumerate}
\item Premières adresses -> adresse de retour
\item Placement des paramètres de la fonction par dessus dans la stack
\item Paramètres locals par dessus dans la stack
\item Libère les paramètres envoyés dans la stack après le return
\end{enumerate}

\subsubsection{Allocation dynamique de la mémoire -> dans le \textbf{HEAP}}
exemple: allocation d'un buffer d'un certain nombre de bytes

\subsubsection{fonction main}
fonction main avec (argc -> compteur d'arguments | argv -> arguments passés à la fonction)
Variable \textbf{ENV} -> contient toutes les variables utilisateur

\section{threads}
Toute les données initialisés et non initialisés (section .data et .text) sont gardées entre les programmes car elles sont les mêmes pour chaque copie du programme

\subsection{Exemple}
Programme pour compter le nombre de places
\begin{lstlisting}
nbplaces = 10
thread1 -> nbplaces - 5
thread2 -> nb places - 3
thread3 -> nbplaces - 4
\end{lstlisting}
\begin{enumerate}
\item nbplaces peut donc être négatif dû à une utilisation en parallèle -> utilisation du \textbf{LOCK} pour la variable.
\item La variable ne pourra pas être touchée par une autre thread pendant le travail d'une autre
\item \textbf{INTERLOCK} -> blockage des variables dont une thread a besoin par une autre thread -> process bloqué
\end{enumerate}

\subsection{Comparaison processus - thread}
Sans threads
\begin{enumerate}
\item 1.5ms recopie intégrale du processus
\item Communication par le \textbf{noyau} -> \textbf{plus lent, nécessite une programmation spécifique}
\end{enumerate}

Avec threads
\begin{enumerate}
\item 0.5ms création par recopie de certains éléments
\item Communication par la \textbf{mémoire commune} -> \textbf{plus rapide, moins d'effort de programmation}
\end{enumerate}
