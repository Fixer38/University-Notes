\chapter{}
\lecture{1}{Lundi 23 Septembre 2019 8:30}{intro}
\section{Structure de l'OS}
\begin{enumerate}
\item Sert a traduire ce que le CPU est capable de faire face à la demande de l'utilisateur.
\item L'ordinateur est capable de faire 3 choses:
\item Structure: Cpu (registres et compteur d'instructions), mémoire (en binaire), sortie INPUT/OUTPUT (périphériques).
\begin{itemize}
  \item Chercher des infos dans la mémoire -> mettre dans un registre
  \item Faire des opérations dans un registre (Comparaisons, arithmétiques)
  \item Renvoyer le résultat dans la mémoire ou vers INPUT/OUTPUT(sorties lentes)
\end{itemize}
\item Les instructions et le contenu de la mémoire sont en binaire
\item l'OS ajoute une couche, interface de plus haut niveau entre les composants machines et l'homme ou entre les programmes et les composants
\end{enumerate}

\section{Assembleur}
\begin{enumerate}
\item Premier language entre le CPU et les composants
\item Était fait en fonctions de l'architecture des CPU
\end{enumerate}

\section{Concurrency}
\begin{enumerate}
\item Les différents programmes doivent se protéger les uns des autres lors de l'exécution (multiutilisateur)
\item Faire attention à la mémoire
\item L'OS fourni cette protection et vérifie les instructions
\end{enumerate}

\section{Prompt}
\begin{enumerate}
\item L'autre façon d'intéragir avec l'OS sans GUI
\item Grâce à une série de commandes (ls, dir, cd)
\end{enumerate}

\section{Fonctions principales}
\begin{enumerate}
\item Machine étendue
\begin{itemize}
  \item Machine virtuelle (exemple: librairie web scraping)
  \item Conviviale et facile d'emploi (GUI)
\end{itemize}

\item Manageur de ressources
\begin{itemize}
  \item Gère les ressources, se protège lui-même.
  \item Première instruction -> Chercher l'OS en lui-même
  \item Partage les ressources de façon cohérente (Empêche la corruption de mémoire entre les programmes)
\end{itemize}
\end{enumerate}
\subsection{Machine étendue}
\begin{enumerate}
\item Opération INPUT/OUPUT 
\begin{itemize}
  \item Création de drivers/pilotes -> Soulage l'OS (exemple: echo, retour à la ligne)
  \item I/O lente (entrée utilisateur -> temps en microsecondes -> milliers d'instructions perdues)
  \item Tout est transformé en binaire
  \item Fait le lien entre l'hardware et l'OS
\end{itemize}
\item Manageur de ressources
Les ressources sont utilisées par des applications et des processus d'arrière-plan.
Il faut les partager:
\begin{itemize}
  \item Partage dans le temps: ordonnancement, le CPU va être utiliser à tour de rôle par les différents processus 
  \begin{itemize}
    \item si attente d'avoir fini -> problème: attente des autres processus.
    \item Temps dédié en millisecondes.
  \end{itemize}
  \item Partage dans l'espace: Gestion de la mémoire -> Mémoire virtuelle avec la RAM et l'HDD
  Passer de la mémoire d'un processus à l'autre = swap -> problème pour ordinateur temps réel(temps défini pour répondre à un processus). 
  Un système de priorité entre les processus est défini dans la file d'attente
\end{itemize}
\end{enumerate}

\section{Classement des OS}
\begin{enumerate}
  \item Mainframe (salle entière)
  \item Serveur (+ petit que mainframe)
  \item Multiprocesseurs (plusieurs CPU en parallèle)
  \item Personnel (Ordinateurs lambdas)
  \item De poche (portable)
  \item Embarqués
  \item Temps réel
\end{enumerate}

\subsection{Mainframe}
\begin{enumerate}
\item Machine capable de générer un nombre énorme d'I/O simultanés
\item Nécessite un OS adapté: Unix
\item 3 types de services:
\begin{itemize}
  \item Batch -> Programme interactif qui tourne tout seul en arrière-plan (backup, calculs)
  \item Transaction processing -> Transaction = suite d'instructions qui n'a de sens que si elle est exécutée complètement (transaction bancaire) -> Incomplète -> Revenir en arrière = rollback
  \item Timesharing -> Partage de temps entre les processus
\end{itemize}
\end{enumerate}

\subsection{Serveur}
\begin{enumerate}
  \item Moins grand et puissant que les Mainframe
  \item Multiutilisateur
  \item Partage de ressources -> hardware et/ou software
\end{enumerate}

\subsection{Multiprocesseurs}
\begin{enumerate}
  \item Séparation en core et multicore
  \item Permet aux ordinateurs de devenir de plus en plus petits
\end{enumerate}

\subsection{Personnel}
\begin{enumerate}
  \item Différents OS vendus avec les PC
  \item Sont énormément plus petits que les serveurs
\end{enumerate}

\subsection{De poche}
\begin{enumerate}
  \item ARM pour les chips plus petites
  \item Arriver aux téléphones
\end{enumerate}

\subsection{Embarqués}
Micropuces, Cartes bancaires, IoT

\subsection{Temps réel}
Machine qui réagit dans un temps bas (programme qui gère les appels téléphoniques -> (carte prépayé, peu de temps pour regarder si la personne est abonnée ou non -> time slice given) number portability (garder le numéro après un changement d'opérateur)
