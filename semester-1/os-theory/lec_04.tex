\lecture{4}{Lundi 21 Octobre 2019 8:30}{Communication inter-processus}
\chapter{Communication inter-processus}
\section{Notion de processus}
Un processus est un programme en cours d ’exécution auquel est associé un
environnement processeur (CO, PSW, RSP, registres généraux) et un environnement
mémoire appelés contexte du processus.

\section{Notion de ressources}
Une ressource logique (variable) ou une ressource matérielle (processeur, périphérique) caractérisée par:
\begin{enumerate}
\item un état: libre/occupée
\item point d'accès (qui peut y toucher) -> un seul processus peut toucher à la variable
\end{enumerate}

\subsection{Allocation des ressources}
Trois étapes:
\begin{enumerate}
\item Allocation
\item Utilisation
\item Restitution
\end{enumerate}

\subsection{interblocage}
2 programmes ont besoin de 2 ressources chacun \\
Une ressource est bloquée par un des 2 programmes -> interblocage -> freeze du système

\subsection{Famine}
Attente infinie -> on ne sait pas quand elle fini

\subsection{Conditions nécessaires à un interblocage}
\begin{enumerate}
\item Besoin de ressources critiques (qui ne peuvent pas être partagées et qui ont un seul point d'accès)
\item Occupation et attente: Processus qui occupe au moins une ressource attend d'acquérir une ressource d'un autre processus
\item Pas de réquisition: Les ressources sont libérées sur une seule volonté des processus les détenant
\item Attente circulaire: Au moins un processus P1 attend une ressource détenue par un autre processus P2, P2 attend lui-même un ressource détenue P1
\end{enumerate}

\subsection{L'exclusion mutuelle}
Les processus disposent d'un espace d'adressage, inacessible par les autres processus
Pour communiquer entre eux les processus utilisent des systèmes de communication tels que les \textbf{pipes}

\subsection{Section critique}
Partie d'un programme dont l'exécution se peut s'entrelacer avec d'autres programmes \\
protection de la variable utilisée par le premier programme pour éviter qu'une thread l'utilise -> \textbf{LOCK}

\subsubsection{Généralisation de ce problème}
Se présente dans:
\begin{enumerate}
\item Base de données
\item Le partage des ressources
\item Les automates
\item Le hardware
\item Le développement
\end{enumerate}

\subsection{Solution matérielle}
Masquage des interruptions

\subsection{Solution algorithmique}
Attente active: Toutes les Xms check de la ressource libre ou non = Polling\\

\section{Solution fournie par le système}
Un sémaphore S peut être vu comme un distributeur de jetons\\
\begin{enumerate}
\item L'opération INIT(S, VAL) fixe le nombre de jetons initial
\item L'opération P(S) attribue un jeton au processus appelant si il en reste sinon -> bloque le processus
\item L'opération V(S) restitue un jeton et débloque un processus de S.L si il en existe un
\end{enumerate}
Jeton = \textbf{mutex} -> mutuellement exclusif un seul point d'accès\\
Avant de lire et de s'allouer une place -> blocage -> Au déblocage du jeton distribue un nouveau jeton

\subsection{Allocation de ressources}
Accès a un ensemble de n ressources critiques\\
Si ensemble de processus -> accès a un ensemble de n ressources critiques\\\\
Allocaton - utilisation - restitution
\begin{enumerate}
\item Allouer une ressource = prendre un jeton Res
\item Rendre une ressource = rendre un jeton Res
\end{enumerate}
Plan d'exécution
\begin{enumerate}
\item Sémaphore Res initialisé à N (1 jeton par ressource)
\item Allocation: P(Res)
\item Utilisation Ressource
\item Restitution V(Res)
\end{enumerate}

\subsection{Lecteur-rédacteur}
\begin{enumerate}
\item Contenu du fichier doit rester cohérent: \textbf{Pas d'écriture simultanée}
\item Lectures doivent être cohérentes: \textbf{Pas de lecture en même temps que les écritures}
\end{enumerate}

\subsection{Producteur-consommateur}
Exemple: Gestion FIFO -> buffer réseau TCP)
\begin{enumerate}
\item L'un produit dans les cases et un autre lit
\item Un producteur ne doit pas produire si le tampon est plein
\item Un consommateur ne doit pas faire de retrait si le tampon est vide
\item Producteur et consommateur ne doivent jamais travailler dans une même case
\end{enumerate}

\subsection{Résumé}
\begin{enumerate}
\item Les exécutions de processus ne sont pas indépendantes: les processus peuvent vouloir communiquer et accéder de manière concurrente à des ressources 
\item Les sémaphores S est un outil système de synchronisation assimilable à un distributeur de jeton et manipulable par seulement trois opérations atomique P(S), V(S) et Init(S)
\item Il existe plusieurs schémas typiques de synchronisation à partir desquels sont élaborés des outils de communication entre processus\\
l'exclusion mutuelle\\
les lecteurs/rédacteurs\\
les producteurs/consommateurs
\end{enumerate}

\section{Les signaux}
Information atomique envoyée à un processus ou à un programme\\
Processus peut réagir de différentes manières à un signal\\
Outils permettant la gestion de processus par l'utilisateur, le système ou autre processus\\
3 différentes réactions d'un processus face à un signal:
\begin{enumerate}
\item Il est dérouté vers une fonction spécifique
\item Le signal est ignoré
\item Peut provoquer l'arrêt du procesus (avec on sans génération d'un fichier core dump)
\end{enumerate}

\subsection{Différents signaux}
\begin{enumerate}
\item SIGINT -> provoqué par le caractère associé à intr sur le clavier de l'utilisateur
\item SIGQUIT -> provoqué par le caractère quit associé au clavier de l'utilisateur
\item SIGILL -> Instruction illégale
\item SIGFPE -> Erreur arithmétique (ex: division par zéro)
\item SIGKILL -> Signal de terminaison, son gestionnaire ne peut pas être remplacé
\item SIGSEGV -> Violation mémoire
\item SIGCHLD -> terminaison d'un fils
\end{enumerate}
