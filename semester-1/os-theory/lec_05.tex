\lecture{5}{Lundi 04 Novembre 2019 8:30}{Interprocessus}

\section{Les tuyaux/pipes}
Utilisé surtout dans les systèmes UNIX -> permet à un groupe de processus de communiquer avec un autre groupe de processus\\
Un coté lit et et un coté écrit -> chaque coté = descripteur de fichier ouvert soit en lecture ou en écriture\\
Le coté de lecture ne peut rien faire tant que l'autre n'a rien écrit -> permet la synchronisation\\
exemple: Lecture des fichiers dans un répertoire = ps -ef | more\\
"more" ne s'exécutera qu'à la fin de la commande précédente, dans ce cas-ci "ps -ef"

\subsection{Problèmes}
Exemple: Plusieurs processus qui lit et un qui écrit: L1, L2 et E1\\
La lecture d'une même donnée n'est pas possible dans ce cas-ci car si L1 la lit, la donnée disparait pour les autres.\\
Besoin du même nombre de pipes de chaque côté.\\
Communication seulement \textbf{unidirectionnelle} et ont besoin d'un \textbf{lien de parenté} (Même utilisateur)

\subsection{Tuyaux nommés}
2 pipes qui n'ont pas de lien de parenté permettent de connecter -> orienté hardware (toujours sur la même machine)\\
exemple: echo coucou > fifo & (envoi de la commande "echo coucou" au process fifo -> rien pour lire "coucou")\\
cat fifo -> echo coucou > fifo (lecteur alloué -> affichage du message "coucou")

\subsection{Sockets}
D'un applicatif à un autre\\
Communication \textbf{bi-directionnelle} et permettant de fonctionner sur \textbf{la même machine}\\
Pour communiquer, l'utilisation d'un socket sur l'une des machines peut être utilisé pour la communication -> Port du socket doit être plus grand que 1023(Well known sockets)\\
Aucun lien de parenté et de pouvoir communiquer vice-versa.


\chapter{Ordonnancement}
\section{Problème}
Quand arrêter un processus?

\section{Objectif}
Maximiser le taux d'occupation du CPU\\
Minimiser le temps de réponse des CPU

\section{Les critères}
Temps d'attente -> somme de tout le temps passé en file prêt\\
Temps de réponse\\
2 mécanismes:
\begin{enumerate}
\item Mécanismes non préemptifs: Qui n'interrompt pas le fonctionnement d'un processus: seul l'attente sur une entrée sortie et la fin d'un processus peuvent provoquer l'appel de l'ordonnanceur
\item Les mécanismes préemptifs: Qui peuvent interrompre le fonctionnement d'un processus, ce qui suppose la mise en place d'un timer
\end{enumerate}
Débit = Throughput: nombre de processus qui s’exécutent complètement dans l’unité
de temps (à maximiser).\\
Temps de rotation = turnaround: le temps pris par le processus de son arrivée à sa
fin (à minimiser).\\
Temps d’attente: attente dans la file prêt (somme de tout le temps passé en file prêt)
(à minimiser).\\
Temps de réponse (pour les systèmes interactifs): le temps entre une demande et la
réponse (à minimiser).

\section{Ordonnanceur et répartiteur}
L'ordonnanceur: Crée la fil d'attente de processus selon des critères bien spécifiques (politique d'ordonnancement) -> envoi des PCB dans la file d'attente(Process control block)\\
Répartiteur: Donne les différents processus aux différents CPU ou coeur d'un même CPU\\
Préemption d'un processus: Fin d'un processus renvoyé à l'ordonnanceur pour être remis dans la file d'attente

\section{Politiques d'ordonnancement}
Premier arrivé, premier servi (FIFO)\\
Plus court d'abord (priorité aux programmes qui prennent le moins de temps)\\
Par priorité constantes (tout le temps la même priorité)\\
Par tourniquet (round robin)\\
Par files de priorités constantes multiniveaux avec ou sans extinction de priorité

\subsection{Premier arrivé, premier servi}
Exemple: Trois processus -> P1(24UT), P2(3UT) et P3(3UT)\\
P1 n'ayant pas de temps d'attente -> s'exécute\\
P2 attend la fin de P1 -> 24UT et s'exécute\\
P3 attend donc 27 UT avant de s'exécuter\\
Si P1 était mis en dernier le temps d'attente est modifié drastiquement\\
\textbf{Le système n'est donc pas prévisible}

\subsection{FSJ: Le plus court d'abord}
Sélection du processus nécessitant le moins de temps d'exécution\\
Méthode non préemptive:
\begin{enumerate}
\item exemple: P1(7UT), P2(4UT), P3(1UT) et P4(4UT)
\item Arrivé de P1 en premier -> s'exécute pendant que P2, P3 et P4 arrivent
\item P3 étant le plus court -> s'exécute
\item P2 s'exécute ensuite et P4 par après
\item T = (3 + 6 + 7) / 4 = 4
\end{enumerate}
Méthode préemptive:
\begin{enumerate}
\item P1 -> s'exécute
\item P2 arrive pendant l'exécution de P1 -> P2 plus court que P1 -> P2 prend le processeur avant que P1 n'ai fini
\item P3 arrive pendant l'exécution de P2 -> P3 plus court que P2 -> P3 s'exécute et se termine
\item P2 ayant déjà une partie de son exécution faite il est donc plus court que P4 -> P2 se termine
\item P4 étant plus court que le reste de l'exécution de P1 -> Fin de P4 et de P1 par la suite
\end{enumerate}
