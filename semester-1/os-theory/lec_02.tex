\chapter{Histoire}
\lecture{2}{Lundi 30 Septembre 2019 8:30}{Histoire des OS}
\section{Concepts incontournables}
\begin{enumerate}
  \item Les processus
  \item L'espace addressable
  \item Le système de fichier -> Organiser le stockage
  \item Les appels système -> Mode utilisateur/système
\end{enumerate}

\subsection{Processus}
\begin{enumerate}
  \item Plusieurs programmes peuvent tournés sur la même machine en même temps en leur allouant des ressources du processeur.
  \item 2 fois le même programme = 2 processus différents.
  \item Daemon = programme en tâche de fond sous manière séquentiel et non pas parallèle (sauf dans le cas d'architecture multiprocesseur)
\end{enumerate}

\subsection{L'espace addressable}
\begin{enumerate}
  \item Les programmes ont besoin de mémoire pour s'exécuter
  \item Elle est limitée -> utilisation du disque en tant que mémoire virtuel (plus de mémoire que ce qu'il y en a)
  \item Grande partie du programme dans la RAM et le reste dans le disque
  \begin{itemize}
    \item si la mémoire RAM est remplie -> envoi de la mémoire la plus vieille dans le disque
    \item Le swap (fait par l'OS) -> prend du temps (machine à temps réel ne peut pas swap)
    \item besoin d'une table pour savoir où se trouve la mémoire dans le disque lié à la mémoire RAM
  \end{itemize}
\end{enumerate}

\subsection{Système de fichiers}
\begin{enumerate}
  \item La racine du disque -> root = /
  \item APFS = système d'Apple
  \item ext2/3/4 = Linux
  \item NTFS, ReFS = Windows
\end{enumerate}

\subsection{Appels système}
\begin{enumerate}
\item Permet au système le contrôle d'accès aux ressources
\item Mode de fonctionnement des processeurs récents:
\begin{itemize}
  \item Mode noyau = accès total (permissions proche de l'OS)
  \item Mode utilisateur = accès restreint
  \item Superviseur = permissions de l'OS
\end{itemize}
\item Programmes: mode utilisateur
\begin{itemize}
  \item Sécurité
  \item Opération à risque -> Appel à l'OS
\end{itemize}
\item Undo = appel système grâce à un registre
\end{enumerate}

\section{Structure des OS}
\begin{enumerate}
\item Différentes manières de conception d'un OS
\begin{itemize}
  \item Monolithiques
  \item En couches
  \item A micro-noyau
  \item Machine virtuelles
\end{itemize}
\end{enumerate}

\subsection{Monolithiques}
\begin{enumerate}
\item Les plus répandus
\item Collection de procédure appelant des routines systèmes
\item L'ensemble de ces routines forme un exécutable le noyau = kernel (d'où les appels systèmes pour passer en mode noyau)
\end{enumerate}
\subsubsection{fonctionnement}
\begin{itemize}
  \item Envoi des paramètres (passer un paramètre à une fonction) -> print("texte") = print(paramètre)
  \item Appel au noyau -> Envoi des paramètres vers la \textbf{stack qui contiendra l'adresse}
  \item Analyse les paramètres 
  \item Sélection de la routine
  \item Exécution de la routine
  \item Retour en mode utilisateur
\end{itemize}

\subsection{En couches}
\begin{enumerate}
\item Le système est construit en couches ayant chacune une fonctionne propre:
\begin{itemize}
  \item Allocation du processeur aux différents processus (multiprogramming)
  \item Gestion de la mémoire
  \item Communication processus-console d'un opérateur (multiusers) -> chaque utilisateur a sa console
  \item Gestion des I/O
  \item Programme des utilisateurs
  \item Opérateur
\end{itemize}
\end{enumerate}

\subsection{Micro-noyau}
\begin{enumerate}
\item Noyaux monolithiques de plus en plus volumineux
\item Appel système courant ou rare dans le noyau
\item Solution:
\begin{itemize}
\item Noyau : Contient les quelques routines courantes
  \item Routines rares : dans des programmes systèmes -> réduit le Nb de lignes dans le noyau
\end{itemize}
\end{enumerate}

\subsection{Machines virtuelles}
\begin{enumerate}
\item Plusieurs machines contenues dans une seule machine
\item IBM 360 -> Un moniteur de machine virtuelle avec un ensemble de machines virtuelles
\end{enumerate}

\section{Historique de systèmes}
Nouvelles intégrations -> nouvelles générations
\subsection{1ère génération}
\begin{enumerate}
\item Pas d'OS
\item 1956: I/O system
\item 1er coprocesseur mathématique
\item Début des supers ordinateurs
\item Mémoire à torre de ferrite
\item Concepteur = programmeur = utilisateur
\item Langage machine
\end{enumerate}

\subsection{2ème génération}
\begin{enumerate}
  \item 1959: IBM 7090: Transistor -> moins de mécanique = moins de panne
  \item 5 fois plus rapide
  \item Plus fiable
  \item Personnel dédié à chaque tâche
  \item Premier ordinateurs personnels
  \item Premiers OS
  \item Programmation à niveau plus élevé (Fortran)
  \item Toujours des périphériques en retard (mémoire perforée) -> télétype
\end{enumerate}

\subsection{3ème génération}
Le circuit intégré -> Toujours plus petit et plus puissant (Jack Kilby et Robett Noyce)

\subsection{Multiutilisateurs}
\begin{enumerate}
\item Compatible time sharing system -> multiutilisateur -> 3 en même temps
\end{enumerate}

\subsection{MULTIX}
Ancêtre d'UNIX

\section{UNIX}
UNplexed Information and Computing Service -> codé en C 
\begin{itemize}
\item Ken Thompson
\item Dennis Retchie
\item Brian Kerrigan
\end{itemize}
\begin{enumerate}
  \item 1970: 1ère version
  \item 1973: Langage C -> Distribution libre
  \item 1975: Version 6 -> Base commune
  \item 1978: BSD -> Berkley university
  \item 1984: System III, IV et V
  \item 1991: Linux
\end{enumerate}

\subsection{POSIX}
Normalisation en 1990 -> Norme POSIX = couche de base avec les mêmes paramètres et la même base.
Passage d'un programme d'un ordi à l'autre. = Portable Operating System Interface

\section{Micros}
\begin{enumerate}
\item Permet de passer aux ordinateurs personnels
\item 1971 Premier microprocesseur = Intel 4004
\item 1975 Premier ordinateur = Altair -> ne sert à rien
\item 1976 Premier Apple (Bojniak)
\item IBM PC: Aout 1981 -> Architecture ouverte -> Appels à des gens spécialisés pour chaque composant et utilisation de MS-DOS
\item Premier programme -> Lotus (tableur pour la gestion)
\item Bill Gates crée son propre OS avec Tim Paterson -> MS-DOS
\item Compétition entre MacOS et MS-DOS
\item 1992: Abandon d'IBM dû à la compétition
\item Retro/Reverse Engenieering = décompilation des programmes
\end{enumerate}

\section{Windows}
\begin{enumerate}
\item Invention de la souris par Kerox, Bill Gates y a cru et à lancer Windows par la suite
\item 1995: Triomphe de Windows sur MacOS
\end{enumerate}
Versions: 
\begin{itemize}
\item 3 branches principales -> Windows 1 à 3.11 -> NT2000 -> 95, 98, Me
\item Fusion XP -> CE -> RT
\end{itemize}
