\chapter{Buts du cours}
\lecture{1}{Jeudi 19 Septembre 2019 13:10}{Intro}
\section{Réflexion}
\begin{enumerate}
  \item Écrire du code propre (clean code)
  \begin{itemize}
    \item Compréhensible
    \item Bien structuré
    \item Avec des noms explicites
    \item Pouvoir réutiliser son code
  \end{itemize}
  \item Réfléchir avant de coder
  \begin{itemize}
    \item Réfléchir à l'algorithmique
    \item Utiliser les instructions les plus adéquates
    \item Choisir des structures de données appropriées
  \end{itemize}
  \item Comprendre ce que vous lisez/écrivez
  \begin{itemize}
    \item Pas d'essais/erreurs
    \item Pas de rustines
    \item Comprendre le "pourquoi" -> déduire correction
  \end{itemize}
  \item Pas de recopier le code d'Internet
  \begin{itemize}
    \item Oublier si incompréhensible
    \item Améliorer le
  \end{itemize}
\end{enumerate}
\section{Gérer votre temps}
\begin{enumerate}
  \item Travailler en dehors des séances d'exercices
  \subsection{Discussion}
  \begin{itemize}
    \item Discuter entre vous et poser des questions
    \item Poser des questions via Discord
  \end{itemize}
\end{enumerate}
\chapter{Module 0: Introduction}
\section{Languages de l'informatique}
\begin{enumerate}
  \item Languages impératifs (procéduraux)
  \begin{itemize}
    \item C
    \item C++ 
    \item C\#
  \end{itemize}
  \item Languages fonctionnels
  \begin{itemize}
    \item F\#
    \item Haskell
    \item Python
  \end{itemize}
  \item Languages logiques
  \begin{itemize}
    \item Prolog
    \item Mercury
  \end{itemize}
  \item Languages multi-paradigmes
  \begin{itemize}
    \item Python
  \end{itemize}
\end{enumerate}
\section{Programmeur et utilisateur}
\begin{enumerate}
  \item Utilisateur: Mot de passe -> accès à un site web
  \item Programmeur: Tache -> Quoi faire -> Comment? -> Algorithme -> Programme
\end{enumerate}
\section{Algorithme}
\begin{enumerate}
  \item Algorithme exact -> Résultat attendu
  \item Algorithme inexact -> Résultat indéfini (fonctionne with edge cases)
  \item Exemples
  \begin{itemize}
    \item Recettes de cuisine
    \item Notice de montage
  \end{itemize}
  \item Définition algorithmique: Sciences des algorithmes qui inclus l'étude de leur complexion et leur conception
\end{enumerate}
\section{Programme}
Séquence d'instructions exécutables par l'ordinateur -> Intermédiaire humain/machine
  \subsection{Schéma exécution programme}
  \subsection{Compiler ou interpréter}
  \begin{itemize}
    \item Compilation: Code source -> Compilation -> Code objet -> Exécution -> Résultat
    \item Interprétation: Code source -> Interprétation -> Résultat
    \item Python: Mélange les 2: Code source -> Compilation -> Code objet -> Interprétation -> Résultat
  \end{itemize}
